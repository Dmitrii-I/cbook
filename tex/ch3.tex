\documentclass[a4paper, 10pt]{article}
\usepackage{listings}
\usepackage[margin=0.5in]{geometry}
\usepackage[usenames,dvipsnames,svgnames,table]{xcolor}
%\usepackage[T1]{fontenc}
\setlength\parindent{0pt}
\lstset{ %
  language=C,                     % the language of the code
  basicstyle=\small,       % the size of the fonts that are used for the code
%  numbers=right,                   % where to put the line-numbers
%  numberstyle=\tiny\color{gray},  % the style that is used for the line-numbers
%  stepnumber=1,                   % the step between two line-numbers. If it's 1, each line 
                                  % will be numbered
%  numbersep=5pt,                  % how far the line-numbers are from the code
  backgroundcolor=\color{white},  % choose the background color. You must add \usepackage{color}
  showspaces=false,               % show spaces adding particular underscores
  showstringspaces=false,         % underline spaces within strings
  showtabs=false,                 % show tabs within strings adding particular underscores
 % frame=single,                   % adds a frame around the code
  rulecolor=\color{black},        % if not set, the frame-color may be changed on line-breaks within not-black text (e.g. comments (green here))
  tabsize=2,                      % sets default tabsize to 2 spaces
  captionpos=b,                   % sets the caption-position to bottom
  breaklines=true,                % sets automatic line breaking
  breakatwhitespace=false,        % sets if automatic breaks should only happen at whitespace
%  title=\lstname,                 % show the filename of files included with \lstinputlisting;
                                  % also try caption instead of title
  keywordstyle=\color{black},     % keyword style
  commentstyle=\color{black},      % comment style
  stringstyle=\color{black},      % string literal style
  escapeinside={\%*}{*)},         % if you want to add LaTeX within your code
  morekeywords={*,...},           % if you want to add more keywords to the set
  deletekeywords={...}            % if you want to delete keywords from the given language
}
\begin{document}

\textbf{Exercise 1} \\
The produced output is ($\bullet$ indicates space): \\
(a) \texttt{$\bullet$$\bullet$$\bullet$$\bullet$86,$\bullet$1040} \\
(b) \texttt{$\bullet$3.02530e+01} \\
(c) \texttt{83.1620} \\
(d) \texttt{1e-06$\bullet$} \\

\textbf{Exercise 2} \\
(a) \lstinline!printf("%-8.1e", x);! \\
(b) \lstinline!printf("%10.6e", x);! \\
(c) \lstinline!printf("%-8.3f", x);! \\
(d) \lstinline!printf("%6.0f", x);! \\

\textbf{Exercise 3} \\
(a) The strings are equivalent. Both will match an input containing an integer number and any number leading and trailing white-space characters. (Recall, that a space in the format string, will match any number of white-space characters in the input, including none (p. 45).) \\

(b) The strings are not equivalent. Using first format string, \texttt{scanf} will look for the minus character immediately following a number. If there is a space before the minus character in the input, e.g. \texttt{2012 -12 -04}, \texttt{scanf} will terminate. The second format string will match input containing any number (inlcuding 0) of spaces before and after the minus characters. \\ 

(c) The strings are equivalent, for the same reason as given in (a). \\

(d) The strings are equivalent. The space after the comma in the second format string is irrelevant, because it is in front of a conversion specification. Any number of spaces will be matched, including none. \\

\textbf{Exercise 4} \\
The values of \texttt{i}, \texttt{x}, and \texttt{j}, will be 10, 0.3, and 5. As the integers numbers do not have decimal fractions, once the dot appears in the input, \texttt{scanf}  moves on to the next item to be scanned.\\

\textbf{Exercise 5} \\
The values of \texttt{x}, \texttt{i}, and \texttt{y}, will be 12.3, 45, and 789. The value of decimal fraction of 45.6 will be purged if it is stored in integer variable \texttt{i}. \\

\textbf{Exercise 6} \\
To adjust the \texttt{addfrac.c} program, simply change the format strings inside \texttt{scanf} function from \lstinline!"%d/%d"! to \lstinline!"%d /%d"!. \\

Note, that no space is needed after the slash because the next item to scan is a conversion specification. Leading and trailing spaces are ignored by \texttt{scanf} when looking for items specified by conversion specifications. Not so for items specified by ordinary characters.\\

\textbf{Project 1} \\
\lstinputlisting{../src/ch3/prj1.c}

\textbf{Project 2} \\
\lstinputlisting{../src/ch3/prj2.c}

\textbf{Project 3} \\
\lstinputlisting{../src/ch3/prj3.c}

\textbf{Project 4} \\
\lstinputlisting{../src/ch3/prj4.c}

\textbf{Project 5} \\
\lstinputlisting{../src/ch3/prj5.c}

\textbf{Project 6} \\
\lstinputlisting{../src/ch3/prj6.c}


\end{document}


