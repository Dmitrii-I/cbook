\documentclass[a4paper, 10pt]{article}
\usepackage{listings}
\usepackage[margin=0.5in]{geometry}
\usepackage[usenames,dvipsnames,svgnames,table]{xcolor}
%\usepackage[T1]{fontenc}
\setlength\parindent{0pt}
\lstset{ %
  language=C,                     % the language of the code
  basicstyle=\small,       % the size of the fonts that are used for the code
%  numbers=right,                   % where to put the line-numbers
%  numberstyle=\tiny\color{gray},  % the style that is used for the line-numbers
%  stepnumber=1,                   % the step between two line-numbers. If it's 1, each line 
                                  % will be numbered
%  numbersep=5pt,                  % how far the line-numbers are from the code
  backgroundcolor=\color{white},  % choose the background color. You must add \usepackage{color}
  showspaces=false,               % show spaces adding particular underscores
  showstringspaces=false,         % underline spaces within strings
  showtabs=false,                 % show tabs within strings adding particular underscores
 % frame=single,                   % adds a frame around the code
  rulecolor=\color{black},        % if not set, the frame-color may be changed on line-breaks within not-black text (e.g. comments (green here))
  tabsize=2,                      % sets default tabsize to 2 spaces
  captionpos=b,                   % sets the caption-position to bottom
  breaklines=true,                % sets automatic line breaking
  breakatwhitespace=false,        % sets if automatic breaks should only happen at whitespace
%  title=\lstname,                 % show the filename of files included with \lstinputlisting;
                                  % also try caption instead of title
  keywordstyle=\color{black},     % keyword style
  commentstyle=\color{black},      % comment style
  stringstyle=\color{black},      % string literal style
  escapeinside={\%*}{*)},         % if you want to add LaTeX within your code
  morekeywords={*,...},           % if you want to add more keywords to the set
  deletekeywords={...}            % if you want to delete keywords from the given language
}
\begin{document}

{\LARGE CHAPTER 5} \\

\textbf{Exercise 1} \\
The produced output is: \\
(a) \texttt{1} \\ 
(b) \texttt{1} \\
(c) \texttt{1} \\
(d) \texttt{0} \\

\textbf{Exercise 2} \\
(a) \texttt{1} \\ 
(b) \texttt{1} \\
(c) \texttt{1} \\
(d) \texttt{1} \\

\textbf{Exercise 3} \\


\textbf{Exercise 4} \\

\textbf{Exercise 5} \\
(a) 3 \\
(b) -3 or 2 \\
(c) -3 or 2 \\
(d) -3 or 2 \\

\textbf{Exercise 6} \\
(a) 3 \\
(b) -3 \\
(c) 3 \\
(d) -3 \\

\textbf{Exercise 7} \\
If the total is a multiple of 10, the answer of the simplified algorithm is different. \\

\textbf{Exercise 8} \\
No, because if the total is a multiple of 10, the check digit will be 10 (which is 2 digits). A check digit is allowed to contain only one digit: 0 through 9.\\

\textbf{Exercise 9} \\
(a) \texttt{63 8} \\
(b) \texttt{3 2 1} \\
(c) \texttt{2 -1 3} \\
(d) \texttt{0 0 0} \\

\textbf{Exercise 10} \\
(a) \texttt{12 12} \\
(b) \texttt{3 4} \\
(c) \texttt{2 8} \\
(d) \texttt{6 9} \\

\textbf{Exercise 11} \\
(a) \texttt{0 2} \\
(b) \texttt{4 11 6} \\
(c) \texttt{0 8 7} \\
(d) \texttt{3 5 4 4} \\

\textbf{Exercise 12} \\
(a) \texttt{6 16} \\
(b) \texttt{6 -7} \\
(c) \texttt{6 23} \\
(d) \texttt{6 15} \\ 

\textbf{Exercise 13} \\
The \texttt{++i} is the same as \texttt{(i += 1)}. The \texttt{i++} is different because when \texttt{--} or \texttt{++} is used as postfix operator, the operand gets incremented or decremented after the expression is evaluated (p. 68). \\

As a double check, run the following program:
\lstinputlisting{../src/ch4/ex13.c}
which will return: \\
\texttt{2 \\ 2 \\ 2} \\

\textbf{Exercise 14} \\
(a) \texttt{(((a * b) - (c * d)) + e)} \\
(b) \texttt{(((a / b) \% c) / d)} \\
(c) \texttt{((((- a) - b) + c) - (+ d))} \\
(d) \texttt{(((a * (- b)) / c) - d)} \\

\textbf{Exercise 15} \\
(a) 3 and 2 \\
(b) 0 and 2 \\
(c) 1 and 2 \\
(d) 1 and 3 \\

\textbf{Project 1} \\
\lstinputlisting{../src/ch4/prj1.c}

\textbf{Project 2} \\
\lstinputlisting{../src/ch4/prj2.c}

\textbf{Project 3} \\
\lstinputlisting{../src/ch4/prj3.c}

\textbf{Project 4} \\
\lstinputlisting{../src/ch4/prj4.c}

\textbf{Project 5} \\
\lstinputlisting{../src/ch4/prj5.c}

\textbf{Project 6} \\
\lstinputlisting{../src/ch4/prj6.c}



\end{document}







































